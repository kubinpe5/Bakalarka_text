\section {MongoDB}
\par MongoDB\cite{mongodb} je multiplatformní NoSQL databáze. Nepoužívá klasické tabulky, založené na relační struktuře, ale má svá vlastní schémata ve formátu BSON\cite{mongogbinc}. Objektová databáze umožňuje rychlejší a~lehčí integraci dat. Tvůrcem této NoSQL databáze je Americká společnost MongoDB Inc.\cite{mongodb} založená v~roce 2007. V roce 2009 přešla společnost MongoDB\cite{mongodb} k open source řešení a~stala se součástí významných společností, například BOSH.
\subsection {Formát BSON}
\par Formát BSON\cite{mongogbinc} je nejvíce využíván při ukládání a~přenosu dat po síti v~neobjektové databázi. Je velmi podobný již známému JSON formátu. Ukládání dat probíhá v~binární formě, což je efektivnější z hlediska rychlosti i~paměťové náročnosti. V některých případech bude BSON zabírat o něco více místa, neboť potřebuje hlavičku, ve které je uložena délka pole s daty.
Například formát JSON bude uložený v~BSON\cite{mongogbinc}u takto: 

\begin{table}[h]
\begin{tabulary}{\textwidth} {| l | l | l |}
 						& \textbackslash{x}16\textbackslash{x}00\textbackslash{x}00\textbackslash{x}00 								&// total document size \\
 						& \textbackslash{x}02 																						&// 0x02 = type String\\
\{"hello": "world"\} 	& hello\textbackslash{x}00 																					&// field name\\
 						& \textbackslash{x}06\textbackslash{x}00\textbackslash{x}00\textbackslash{x}00world\textbackslash{x}00 		&// field value\\
 						& \textbackslash{x}00 																						&// 0x00 = type EOO \\
\end{tabulary}
\caption[MongoDB příklad formátu BSON]{MongoDB příklad formátu BSON}\label{tab:bson}
\end{table}
\vfill