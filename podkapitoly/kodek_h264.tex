\section {Kodek VP8 a ukládání videí v~NARRA}
\par Pro správné pochopení významu kodeku VP8\cite{vp8} je potřeba trocha teorie k WebM\cite{webm}. WebM je otevřený formát multimediálních souborů používaný na webu. WebM soubory se skládají z obrazových toků komprimovaných právě kodekem VP8, nebo VP9. V projektu NARRA je použita komprimace pomocí VP8. V aktuálním nastavení se videa počítají do formátu WebM (video v~kodeku VP8, audio ve Vorbis) ve dvou kvalitách: 720p s bitrate 1Mbps a 180p s bitrate 300kbps. Navíc je počítán čistě zvukový náhled ve Vorbis (kontejner OGG) a pět náhledů v~rozlišení 350x250 ve formátu PNG. 
\par Pro uložení videa se používá kontejner (například WebM, MP4, AVI, MOV, \ldots), tyto formáty definují vnitřní strukturu a složení jednotlivých datových proudů do výsledného souboru. Tento datový proud je potřeba zkomprimovat. K účelu komprimace slouží právě kodek (MPEG, VP8, H.264, \ldots). Komprimace slouží pro zmenšení datového toku pro záznam videa.
\subsection{VP8}
\par Jak již bylo zmíněno VP8 je formát pro kompresi dat vlastněný společností Google. Je založen na knihovně \texttt{libvpx}, která jediná umí zakódovat VP8 video stream. Dekódování probíhá také pomocí Google knihovny \texttt{libvpx}.