\section{Projekt Narra}
\par Narra\cite{narra} je~projekt s volně dostupným zdrojovým kódem, který se~zabývá anotací a~propojením audiovizuálních médií a~textu. Podobně jako YouTube má dostupné API a~po dokončení bude sloužit umělcům, filmařům a~dalším, kteří chtějí vytvářet otevřený příběh (Open Narrative) nebo editovat videa. Dokumentaristé mohou tento projekt využít pro rozsáhlejší díla, díky nabídnutému relevantnímu obsahu s metadaty. Projekt zastřešuje FAMU CAS a~celý vývojářský tým tvoří 5 lidí včetně studentů dokončujících Bakalářské a~Magisterské studium.
\par S první myšlenkou projektu Narra přišel v~roce 2002 - 2003 Eric Rosenzveig a~Willy LeMaitre spolu s dalšími mediálními umělci a~programátory. Dílo bylo rozdělené na tři části. První\uv{playListNetWork} byl opensource software vyvinutý na základě konzultací s umělci o audiovizuálním obsahu a~rozhraním pro vizualizaci. Umožňoval práci více uživatelů na různých místech a~pomocí textových poznámek upravovat popisky skladeb a~videí. Druhý\uv{disPlayList} bylo veřejně přístupné rozhraní pro streamování medií z~playListNetWork. Jednalo se~o webovou aplikaci, která vizualizovala výsledná videa do grafu, ze kterého šel pomocí klíčových slov tvořit další celek.\uv{Ressemblage}, neboli poslední část, byl výsledkem práce umělců používajících novou technologii práce s médii.
\par V projekt Open Narrative se~zapříčinil nejvíce umělec Eric Rosenzveig a~redaktor Tomáš Dobruška na FAMU v~letech 2010 - 2015. Díky penězům z~grantu mohou pokračovat ve vývoji spolu s KSI FIT ČVUT.

\section{Technologie v~projektu Narra} 
\par Narra je~psaná v~jazyce Ruby a~poskytuje REST-API pro komunikaci se~světem. Další použité technologie jsou Sidekiq, OmniAuth, MongoDB a~Rails. Všechny tyto komponenty zajišťují stabilní jádro aplikace, na které je~možné navazovat dalšími balíčky (gem) jako v~případě mé bakalářské práce. Pro začátek jsem se~musel seznámit s doménovým modelem\cite{michalmoc} celé aplikace, který je~na~obrázku 2.1.

\begin{figure}[H]
\includegraphics[width=1\textwidth]{./obrazova_priloha/domain_model.png}
\caption{Doménový model celého projektu Narra}
\end{figure}

\par User reprezentuje přihlášeného uživatele, který chce používat aplikaci. Každý uživatel může mít vazbu na projekt a~knihovnu. Entita Item obsahuje informace o jménech souboru, url videa a~vlastních stažených souborů. Dále obsahuje vazby na knihovnu, ve které je~uložen sám Item. Itemy se~podle definice v~modelu nemohou vyskytovat samostatně, musí být organizovány v~knihovnách. V případě mého projektu nebudu vytvářet Item ale jeho potomka Video. 
\par Další vazba je~na entity MetaItem, které reprezentují generátor pro metadata. MetaItem je~třídní potomek Meta, který přebírá atributy rodiče, jež jsou povinné. Při ukládání položek v~Meta musí být vyplněno Meta.name i~Meta.value. Po prozkoumání doménového modelu NARRA jsem vytvořil můj model viz obrázek 2.2.

\begin{figure}[H]
\includegraphics[width=1\textwidth]{./obrazova_priloha/domain_my.png}
\caption{Doménový model mé části Narry}
\end{figure}
\par Moje část aplikace se~týká především entit Video, MetaItem a~Library. Entita Library reprezentuje celou knihovnu videí, se~kterou se~bude lokálně pracovat. Video (potomek Item) je~entita, které budu při vytváření asistovat poskytnutím URL při zadávání. Po zadání budu muset zpracovat obsah odkazu a~vrátit metadata a~cestu ke stažení videa. Jméno pro konkrétní prvek je~reprezentováno řetězcem a~url je~také řetězec. Pro jednoduché volání budu mít vytvořenou třídu Connector, potomka Narra::SPI, která bude provádět inicializace, validaci, popis metadaty a~stažení videa a~případných titulků.
\par Zpracování videa probíhá vytvořením nového prvku (Item) v~knihovně. Pro zpracování se~použije POST požadavek s parametry v1/items/new (author, admin). Po zpracování dostaneme strukturu nově vytvořeného prvku (Item).
\par Příklad POST požadavku\cite{narra_en}:
\begin{verbatim} 
POST v1/items/new
url: "http://example.org/00329O-051.mov"
library: "552a328961633276b1000000" 
author: "Camera Guy"
metadata: {"description": "Some interesting description"}
\end{verbatim}
\hfill
\par Příklad odpovědi:
\begin{verbatim}
{"status":"OK","item":{
  "id":"552a338961633277b1000000",
  "name":"00329O-051",
  "url":"http://example.org/00329O-051.mov",
  "type":"video",
  "prepared":false,
  "library":{"id":"552a...b1000000","name":"Example Library"},
  "metadata":[
   {"name":"type","value":"video","generator":"source"},
   {"name":"name","value":"00329O-051","generator":"source"},
   {"name":"url","value":"url","generator":"source"},
   {"name":"library","value":"test","generator":"source"},
   {"name":"author","value":"testovaci","generator":"source"},
   {"name":"description","value":"test","generator":"bob"}
  ]
}}
\end{verbatim}
\par V odpovědi jsem musel zkrátit identifikátor knihovny, aby se~mi vešel výraz na stránku. Ze stejného důvodu jsem byl nucen zkrátit hodnotu url. Zde je~možné nastavit, zda bude projekt veřejný. V takovém případě je~možné dostat přístup pouze pro čtení za předpokladu, že je~knihovna součástí projektu. Pro vyšší práva je~nutností být contributor (spolupracovník) daného projektu. Získání výpisu přístupných knihoven se~provádí GET požadavkem na zdroj/adresu v1/libraries(author/admin).