\section {Testovací nástroj RSpec}
\subsection{Historie}
\par RSpec vznikl jako experiment Stevena Bakera, Davida Astelse a Aslaka Hellesøe. S vývojem začali v roce 2006, na Ruby on Rails. První vydaná verze 1.0 vyšla o rok později a obsahovala funkce, které má RSpec dodnes. Měl ovšem několik ne úplně efektivně naimplementovaných částí a proto musel být přepsán.
\par Koncem roku 2008 zabudoval Chad Humphries "Micronauta", který v sobě obsahoval systém metadat a poskytoval mnohem lepší flexibilitu, než RSpec 1.0. V roce 2010 začali David a Chad pracovat na verzi 2. Chtěli celý projekt rozdělit do mudulů, které by pak mohli pužívat samostatně a navazovat jedním na druhý. Jako jádro posloužil Micronaut, na který navazovali moduly.
\par Listopad roku 2012 byl pro vývor RSpecu zlomový. David Asteles se rozhodl od projektu odpojit a věnovat se jiným věcem. Lídrem pro RSpec se stal Myron Marston a pro rspec-rails byl nominován Andy Linderman. Nově vytvořený tým začal pracovat na verzi RSpec 3, která byla spojením a vyčištěním všech předchozích verzí dohromady. Po vydání RSpec 3 odešel Andy Linderman do důchodu. Dodnes se RSpec rozvíjí díky velké komunitě spolupracovníků.
\par V mé Bakalářské práci budu pracovat s jádrem testovacího nástroje RSpec a očekáváními, neboli expectations. V následujících dvou podkapitolách uvedu postup instalace a ukázky použití daných komponent.

\subsection{RSpec core}
// TODO psát nebo nepsat

\subsection{RSpec expectations}
// TODO psát nebo nepsat
