\section {Testovací nástroj RSpec}
\subsection{Historie}
\par RSpec vznikl jako experiment Stevena Bakera, Davida Astelse a Aslaka Hellesøe. S vývojem začali v roce 2006, na Ruby on Rails. První vydaná verze 1.0 vyšla o rok později a obsahovala funkce, které má RSpec dodnes. Měl ovšem několik ne úplně efektivně naimplementovaných částí a proto musel být přepsán.
\par Koncem roku 2008 zabudoval Chad Humphries "Micronauta", který v sobě obsahoval systém metadat a poskytoval mnohem lepší flexibilitu, než RSpec 1.0. V roce 2010 začali David a Chad pracovat na verzi 2. Chtěli celý projekt rozdělit do mudulů, které by pak mohli pužívat samostatně a navazovat jedním na druhý. Jako jádro posloužil Micronaut, na který navazovali moduly.
\par Listopad roku 2012 byl pro vývor RSpecu zlomový. David Asteles se rozhodl od projektu odpojit a věnovat se jiným věcem. Lídrem pro RSpec se stal Myron Marston a pro rspec-rails byl nominován Andy Linderman. Nově vytvořený tým začal pracovat na verzi RSpec 3, která byla spojením a vyčištěním všech předchozích verzí dohromady. Po vydání RSpec 3 odešel Andy Linderman do důchodu. Dodnes se RSpec rozvíjí díky velké komunitě spolupracovníků.
\par V mé Bakalářské práci budu pracovat s jádrem testovacího nástroje RSpec a očekáváními, neboli expectations. V následujících dvou podkapitolách uvedu postup instalace a ukázky použití daných komponent.

\subsection{RSpec core}
\par RSpec core je samotné jádro aplikace, na které navazují další moduly. Je nezbytnou částí pro testování. Instalace je šložená ze tří příkazů \textit{gem install rspec}; \textit{gem install rspec-core}; \textit{rspec --help} pro nápovědu k nově nainstalovanému softwaru. První příkaz nainstaluje rspec-core, rspec-expectations a rspec-mocks, což je kompletní balíček pro testování. Druhý příkaz nainstaluje pouze rspec-core, který neobsahuje všechny funkčnosti.
\par Základní struktura testování je velmi podobná hovoru v angličtině. Používají se slova "describe" a "it", která mají stejný význam jako v mluveném slově.\\
"Describe an order."\\
"It sums the prices of its line items."\\
\begin{minted}{ruby}
RSpec.describe Order do
  it "sums the prices of its line items" do
    order = Order.new
    order.add\_entry(LineItem.new(:item =\textgreater Item.new(
      :price =\textgreater Money.new(1.11, :USD)
    )))
    order.add\_entry(LineItem.new(:item =\textgreater Item.new(
      :price =\textgreater Money.new(2.22, :USD),
      :quantity =\textgreater 2
    )))
    expect(order.total).to eq(Money.new(5.55, :USD))
  end
end
\end{minted}
%\end{displayquote}

\subsection{RSpec expectations}
// TODO psát nebo nepsat
