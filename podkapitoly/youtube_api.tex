\section{YouTube API}
\par YouTube Aplication Programming Interface\cite{apistart} je nástroj pro vývojáře, který umožňuje snadný přístup ke statistikám a~datům na konkrétním YouTube kanálu a i k jednotlivým videím. API je rozdělené na tři hlavní části:
\begin{itemize}
	\item{Players and Player APIs, které umožňují uživatelům sledování videí ve vaší aplikaci a~zjištění zpětné vazby od uživatelů.}
	\item{Data and Analytics APIs, zběžně popisuje rozhraní pro přístup k~funkcím a~datům uloženým v databázi YouTube.}
	\item{Buttons, Widgets, and Tools slouží k~popisu všech nástrojů, které může vývojář použít pro svou aplikaci.}
\end{itemize}

\section{Data and Analytics API}
\subsection{YouTube Data API (v3)}
\par Pro mou práci potřebuji druhou část Data and Analytics APIs\cite{apistart}\cite{apiv3}. Tato část rozhraní umožňuje začlenění informací YouTube do vlastní aplikace. Mám v plánu jí použít pro získání metadat, kde HTTP/GET požadavkem získám JSON objekt od YouTube API, s nímž budu nadále pracovat. Pro zahájení práce s API je potřeba si založit Google účet a~získat vlastní API klíč. Vytvoření projektu a~API klíče popíšu podrobněji v praktické části práce.
\par 
\par API vyžaduje v požadavku seznam dílčích zdrojů, které si přejeme získat, aby se zabránilo zbytečnému přenosu dat a~nebyla přetěžována síť ani procesor. YouTube API obsahuje omezení denní kvótou, o které se dozvíte více v následující kapitole. Tento přístup zajišťuje efektivní práci a~využití prostředků, proto jsou požadavky rozděleny na části.
\begin{itemize} 
\item{snippet}
\item{contentDetails}
\item{fileDetails}
\item{player}
\item{processingDetails}
\item{recordingDetails}
\item{statistics}
\item{status}
\item{suggestions}
\item{topicDetails}
\end{itemize}
\par Díky rozkouskování do částí je možné se dotázat na specifické informace a~ušetřit si tak svou denní kvótu. V návaznosti dojde ke snížení latence a~aplikace bude rychlejší. Pro správnou funkčnosto požadavků je potřeba mít vlastní API klíč. Projekt NARRA zatím nemá získávání API klíče funkční nicméně je plánované automatické získávání API klíčů pro každého uživatele s Google identitou v systému NARRA.

\subsection{Omezení}
\par Každá aplikace má určitá omezení paměti\cite{quota}\cite{googleconsole}, či časového kvanta, které uživateli přidělí. YouTube omezuje pomocí kvót počet požadavků, ve kterých měří využití výpočetního výkonu pro jednotlivé uživatele. Podporovány jsou čtyři typy operací.
\begin{itemize}
\item {list}
\item {insert}
\item {update}
\item {delete}
\end{itemize}
\par Operace list vrátí GET požadavek spolu žádným či více výsledky. Insert vytvoří pomocí pažadavku POST nový prostředek. Update změní již existující prostředek a~nahradí ho novým. Poslední delete vymaže existující prostředek, který jsme specifikovali. Operace insert, update a~delete vyžadují autorizaci uživatele, nejčastěji pomocí jeho vlastního API\cite{apistart} klíče. Operace list funguje i~v~případech bez autorizace, tak s autorizací.
\par Další důvod pro používání kvót je zajištění jisté úrovně efektivity u vývojářů softwaru používajících Data API, a~ne vytvářet aplikace, které omezují ostatní a~snižují tak kvalitu poskytovaného softwaru. Výše jednotlivých kvót se pro požadavky liší podle náročnosti operací, která se má provést. Jsou zde dva základní faktory, které ovlivňují z~většiny cenu požadavku. Při načtení ID z~videa zaplatíte 1 jednotku. Operace zápisu stojí přibližně 50 jednotek. Nejdražší je nahrání videa, které se pohybuje okolo 1~600 jednotek za jedno video. To se Vám může zdát jako velmi vysoká cena, ale není tomu tak. 
\par Operace čtení a~zápisu nemají přesně stanovenou kvótu, neboť mohou číst a~zapisovat odlišné množství částí videí. Pro tento účel YouTube API vytvořilo několik odlišných kategorií, aby umožnilo využít pouze nezbytně malou část kvóty pro požadavek. Po tomto krátkém úvodu se dostáváme k~přidělenému počtu jednotek pro jedno aplikaci. Každá aplikace dostane 50~000~000 jednotek na den, což odpovídá přibližně 1~000~000 operací čtení, kde má každý zdroj dvě části, nebo 50~000 operací zápisu a~450~000 dalších operací čtení, kde má každý zdroj znovu dvě části. Poslední příklad je přibližně 2~000 nahraných videí, 7~000 operací zápisu a~200~000 operací čtení, kde má každý zdroj tři části.
\par Předchozí odstavec byl jen letmý příklad, kolik si YouTube API účtuje jednotek za své služby. Pro detailnější informace je potřeba nahlédnout do své Google konzole na adrese \url{https://developers.google.com/youtube/v3/determine_quota_cost}\cite{quota}. Zde jde velmi snadno zjistit konkrétní cenu požadavku aplikace pomocí připravené tabulky. Předběžně vypočítané cena u mé aplikace je 9 jednotek na jeden požadavek, což znamená více jak pět milionů zpracovaných videí za jeden den. To je více než dostačující kapacita a~proto nemusím omezovat metadata, která v~mé aplikaci plánuji z YouTube získávat. 