\section{Vyrovnávací paměť médií}
\subsection{Obecný popis}
\par Vyrovnávací paměť slouží ke zrychlení systému pomocí \uv{nakešování} informací, které by byly čteny například z pevného disku. Zde je zrychlení docíleno díky různým přístupovým časům mezi diskem a vyrovnávací pamětí. Přítup na disk trvá okolo 10ms, zatímco čtení z cashe je otázkou 10ns. Velmi rychlou matematikou se dá vypočítat, že je cashe 1000x rychlejší v porovnání s pevným diskem, proto je potřeba uložit často používané informace přechodně do vyrovnávací paměti. Mezi nevýhody pamětí cashe patří jejich velikost. Zatímco na disk jsme v dnešní době schopni uložit řádově TB, do vyrovnávací paměti lze dostat pouze data řádu MB.
\par V projektu NARRA je také potřeba pracovat s vyrovnávací pamětí médií, neboť je zde vytěžován server, datové linky a další komponenty projektu. Řešení spočívá ve vytvoření jednorázového média v náhledové kvalitě. Náhledová kvalita postačuje pro zjištění obsahu videa a zároveň rychlou práci pro střih, zatímco po dokončení práce se již vyšlou z projektu příslušné požadavky a zajistí celé sestříhané video v nejvyšší možné kvalitě.
\par Uložení náhledového videa má i další důvody: uživateli je možné poskytnout takové zdroje, ke kterým nemá přímo přístup a dále se video v náhledové kvalitě ztratí v případě, že YouTube původní video zablokuje. Tento jev se děje velmi často a je spojen s porušováním autorských práv, proto je podstatné mít multimédium ve vyrovnávací paměti, neboť není dostupné jako soubor a tak by bylo třeba ho při každém požadavku znova stahovat a zpracovávat.

\subsection{Realizace v projektu NARRA}
\par // todo

1) Tvůj konektor poskytne informaci kde se nachází soubor s multimediálním obsahem.
2) Pracovní server NARRA navštíví tuto adresu (kterou hostuji na tom download serveru), čímž dojde ke stažení videa, uložení na cestu dotupnou přes http a zaslání hlavičky 303 s lokací souboru.
3) Pracovní server tedy náseduje přesměrování. (k timeoutu nedochází)
4) Pracovní server přepočítá videosoubor do všech formátů potřebných v NARRA (WebM ve vysoké a nízké kvalitě; zvukový soubor ve formátu OGG Vorbis) odkazy na tyto soubory sám předá do databáze NARRA

\note{V NARRA je to tak, že při vytváření entity Item se z konektoru který umí danou URL obsloužit (existují uvnitř zabudované konektory pro multimediální soubory dostupné přímo přes HTTP) vytáhnou všechny potřebné informace včetně adresy pro stažení fyzického souboru s multimédiem. Jako poslední krok po uložení Item se spustí processing (narra-core/lib/narra/core/items.rb:88), což v podstatě znamená, že se do fronty úkolů v systému SideKick zařadí úloha překódování videa do požadovaných formátů (definováno v nastavení instance, viz dotaz na API v1/settings, 1.5.2 v dokumentaci API).

Dobré je pak říct, že pro tvoje účely jsi musel vytvořit i způsob jak video z YouTube stáhnout a poskytnout dočasně jako soubor. Vše je to postavené na serveru nginx, skript pro stažení a poskytnutí vypadá takto (nicméně pozor: tohle předpokládá, že ID je v pořádku (což by mělo) ale ošetřené by to být mělo skript asi během zítřka ještě upravím):}

\begin{minted}{python}
#!/usr/bin/env python
# -*- coding: utf-8 -*-

import web, subprocess, os

urls = ("/.*", "youtube")
app = web.application(urls, globals())

class youtube:
  def GET(self):
    data = web.input(id="0")
    video = data.id
    if video == "0":
       raise web.notfound()
 
    filename = subprocess.check_output(['youtube-dl',
        '--get-filename','-o','"%(id)s.%(ext)s"',video])
    filename = filename.strip('"\n')
    
    os.system("youtube-dl -o /data/%s %s >/dev/null" % (filename, video))
 
    raise web.seeother('/'+filename)
 
if __name__ == "__main__":
  web.wsgi.runwsgi = lambda func, addr=None: web.wsgi.runfcgi(func, addr)
  app.run()

\end{minted}