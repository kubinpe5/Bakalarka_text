\section{Vyrovnávací paměť médií}
\subsection{Obecný popis}
\par Vyrovnávací paměť\cite{sap} slouží ke zrychlení systému pomocí \uv{nakešování} informací, které by byly čteny například ze vzdáleného zdroje. Zde je zrychlení docíleno díky různým přístupovým časům mezi vzdáleným diskem a lokálním diskem. Přístup na vzdálený disk disk trvá řádově vteřiny, zatímco čtení z lokálního disku trvá o poznání kratší dobu. Velmi rychlou matematikou se dá vypočítat, že je cashe několik řádů rychlejší v porovnání se vzdáleným úložištěm, proto je potřeba uložit často používané informace přechodně do vyrovnávací paměti. Mezi nevýhody pamětí cashe patří jejich velikost. Zatímco vzdálený disk, či claudové úložiště je řádově TB (v případě YouTube nemá šanci jeho velikost smrtelník odhadnout), lokálního úložiště je řádově GB \verb|~| TB a je cenově nákladnější.
\par V projektu NARRA je také potřeba pracovat s vyrovnávací pamětí médií, neboť je zde vytěžován server, datové linky a další komponenty projektu. Řešení spočívá ve vytvoření jednorázového média v náhledové kvalitě. Náhledová kvalita postačuje pro zjištění obsahu videa a zároveň rychlou práci pro střih, zatímco po dokončení práce se již vyšlou z projektu příslušné požadavky a zajistí celé sestříhané video v nejvyšší možné kvalitě.
\par Uložení náhledového videa má i další důvody: uživateli je možné poskytnout takové zdroje, ke kterým nemá přímo přístup a dále se video v náhledové kvalitě neztratí v případě, že YouTube původní video zablokuje, případně bude odstraněno původním vlastníkem. Tento jev se děje velmi často a může být spojen s porušováním autorských práv. V případě YouTube je extrémně důležité mít multimédium ve vyrovnávací paměti, neboť není dostupné jako soubor a tak by bylo třeba ho při každém požadavku znova stahovat a zpracovávat.

\subsection{Realizace v projektu NARRA}
\par Výsledná realizace spočívá v těchto krocích:

\begin{itemize}
\item Můj konektor poskytne informaci, kde se nachází soubor s multimediálním obsahem.
\item Pracovní server NARRA navštíví tuto adresu (hostovanou na serveru \url{http://neptun.avc-cvut.cz/}), čímž dojde ke stažení videa, uložení na cestu dotupnou přes http a zaslání hlavičky 303 s lokací souboru.
\item Pracovní server tedy náseduje přesměrování. %(k timeoutu nedochází)
\item Pracovní server přepočítá videosoubor do všech formátů potřebných v NARRA (WebM ve vysoké a nízké kvalitě; zvukový soubor ve formátu OGG Vorbis). Odkazy na tyto soubory sám předá do databáze NARRA.
\end{itemize}

\par Uvnitř NARRY se při vytvoření entity Item vytáhnou z konektoru, který umí danou URL obsloužit (existují uvnitř zabudované konektory pro multimediální soubory dostupné přímo přes HTTP), všechny potřebné informace včetně adresy pro stažení fyzického multimediálního souboru. Jako poslední krok po uložení Itemu dojde ke spuštění zpracování (narra-core/lib/narra/core/\newline items.rb:88). To znamená, že se do fronty úkolů v systému SideKick zařadí úloha překódování videa do požadovaných formátů (definováno v nastavení instance, viz dotaz na API v1/settings, 1.5.2 v dokumentaci API).
\par Dále jsem musel pro moje účely vytvořit způsob jak stáhnout a dočasně poskytnout YouTube video jako soubor. Vše je postaveno na serveru nginx. Tento kód je pouze návrhem kódu, který poběží na serveru. Podílel jsem se pouze na myšlence tohoto kódu, nikoli na skutečné implementaci. Implementací se zabývali programátoři z projektu NARRA. Z mého konektoru jsem pouze musel zajistit, že bude identifikátor videa v pořádku.

\begin{minted}{python}
#!/usr/bin/env python
# -*- coding: utf-8 -*-

import web, subprocess, os

urls = ("/.*", "youtube")
app = web.application(urls, globals())

class youtube:
  def GET(self):
    data = web.input(id="0")
    video = data.id
    if video == "0":
       raise web.notfound()
 
    filename = subprocess.check_output(['youtube-dl',
        '--get-filename','-o','"%(id)s.%(ext)s"',video])
    filename = filename.strip('"\n')
    
    os.system("youtube-dl -o /data/%s %s >/dev/null" % (filename, video))
 
    raise web.seeother('/'+filename)
 
if __name__ == "__main__":
  web.wsgi.runwsgi = lambda func, addr=None: web.wsgi.runfcgi(func, addr)
  app.run()
\end{minted}