\section{DublinCore}
\par DublinCore\cite{dublincoredocementation} je soubor metadatových prvků, jehož cílem je umožnit rychlé a snadné vyhledávání v elektronických zdrojích. Původně byl vytvořen jako popis metadat v webových stránek, postupně zaujal vyšší instituce a experty z různých odvětví, například muzeí, kni\~ho\~ven a dalších komerčních ogranizací. Vedení sídlí ve státě Ohio v Severní Americe.

\subsection{Historie DublinCore}
\par První setkání tvůrců DublinCore\cite{dublincoredocementation} bylo v březnu 1995. Jejich cílem bylo popsat elektronická data, na základě sémantických pravidel a byla zde uvedena problematika vyhledávání v elektronických dokumentech.

\par Druhý seminář se konal o rok později v dubnu 1996. Tvůrce tentokrát hostilo město Warwick ve Velké Británii. Po přiblížení problematiky mezinárodní komunitě se zaměřili na problém syntaxe a sémantiky, kterou by byly schopné efektivně zpracovat Webové aplikace. Rychlé zavedení technologie DublinCore do Webových aplikací vedlo k rychlému rozšíření této metody do světa. Na tomto setkání byl vytvořen základ architektury metadat, Warwick Framework.

\par V září 1996 byl další seminář přesunut do Spojených států amerických do města Dublin. Hostil převážně experty na grafiku, kteří spolu s tvůrci DublinCore diskutovali o spojení mezi vizuální a textovou částí metadat. Měli za cíl spojit požadavky DublinCore a Warwick Frameworku.

\par Další krok ve vývoji se odehrál v roce 1997 ve městě Canbery v Austrálii. Hlavní myšlenou bylo učinit popis více minimalistický a lépe ho strukturovat. Šlo o zjednodušení a upravení popisu pro následná další rozšíření, neboť každý jistě ví, že na hliněných nohách se kvalitní barák postavit nedá. Proto celé specifikaci dali stejnou strukturu a učinili jí minimalistickou.

\par Další semináře se konali v říjnu 1997 v Helsinkách, kde byl kladen důraz na datum, pusobnost a vztah. Ve Washingtonu v roce 1998 bylo svolané setkání, pro sjednocení různých implementací DublinCore. Další seminář byl rekordní v počtu odborníků. Sešlo se jich 120 z 27 zemí. Další semináře už nastíním pouze pomocí tabulky.

\begin{table}[h!]
\centering
\caption[DublinCore historie]{Schůzky DublinCore mezi lety 2000 - 2014}\label{tab:dublincore}
\begin{tabular}{| l | l | l |}
\hline
Rok & Město & Země\\
\hline
2000 & Otawa & Kanada\\
\hline
2001 & Tokyo & Japonsko\\
\hline
2002 & Florencie & Itálie\\
\hline
2003 & Seatle & Washington, USA\\
\hline
2004 & Sanghai & Čína\\
\hline
2005 & Madrid & Španělsko\\
\hline
2006 & Manzanillo, Colima & Mexico\\
\hline
2007 & Singapur & Singapur\\
\hline
2008 & Berlin & Německo\\
\hline
2009 & Soul & Jižní Korea\\
\hline
2010 & Pittsburgh & Pennsylvania, USA\\
\hline
2011 & Haag & Nizozemí\\
\hline
2012 & Kuching, Sarawak & Malaysie\\
\hline
2013 & Lisabon & Portugalsko\\
\hline
2014 & Austin & Texas, USA\\
\hline
\end{tabular}
\end{table}

\subsection{Struktura DublinCore}
\par Pro mou bakalářskou práci jsem zvolil popis sekci 3 z \url{http://dublincore.org/documents/dcmi-terms/}, kterou jsem ještě upravil, abych využil všechna metadata co mi youtube nabízí a zároveň se zbavil duplicit, které se v této sekci nacházejí. Veškeré popisy jsou dostupné na \url{http://dublincore.org/documents/dcmi-terms/}.
\par Struktura objektu s daty o youtube videu se lehce lišila od stanovené struktury z DublinCore\cite{dublincoredocementation}. V následující tabulce je náhled na originální změní pořadavků, které jsem musel upravit. Více o této úpravě je napsáno v části Realizace, kapitola Popis metadaty.
\hfill\vfill
\begin{table}[h!]
\caption[DublinCore metadata]{Ujednocená struktura metadat}\label{tab:dublincore1}
\begin{tabular}{|p{.20\textwidth} | p{.40\textwidth} | p{.40\textwidth}|}
\hline
Popisek & Definice & Komentář\\
\hline
Přispěvatel (contributor) & Subjekt zodpovědný za zdroj příspěvku. & Například jméno osoby, organizace, nebo služby.\\
\hline
Krytí (coverage) & Prostorová a časová použitelnost zdroje. & Například označení místa geografickými souřadnice mi, nebo lhůta či časové období.\\
\hline
Tvůrce (creator) & Subjekt zodpovědný za zpřístupnění zdroje. & Napříkald jméno osoby, organizace, nebo služby.\\
\hline
Datum (date) & Doba spojená s událostí v životním cyklu zdroje. & Vyjádření časové informace. \\
\hline
Popis (describtion) & Popis dané entity. & Popis může být textový nebo grafický. \\
\hline
Formát (format) & Formát souboru, fyzický nosič, nebo rozměry zdroje. & Například délka trvání videa. \\
\hline
Identifikátor (identifier) & Primářní klíč zdroje. & Řetězec maximálně dvanácti znaků mezi ?v= a \&, nebo koncem URL.\\
\hline
Jazyk & Jazyk zdroje. & YouTube API neposkytuje přímý přístup k jazyku, proto ho nebudu do mých metadat používat.\\
\hline
Vydavatel & To samé co Tvůrce. & Zde mají v metadatech duplicitu. \\
\hline
Vztah (relation) & Související zdroje. & V mám případě se jedná o categoryId. \\
\hline
Práva (rights) & Informace o právech obsahu. & Tento atribut je v YouTubeAPI popsán dvěmi entitami. LicensedContent a License.\\
\hline
Zdroj (source) & Přibuzný zdroj od kterého je odvozen popsaný zdroj. & Tento atribut v mých metadatech mít nebudu.\\
\hline
Předmět (subject) & Téma zdroje. & Toto je znovu entita, kterou YouTubeAPI neobsahuje.\\
\hline
Název (title) & Jméno dané zdroji. & Atribut title, který vracím samostatně pro lěpší práci s následným obsahem.\\
\hline
Typ (type) & Povaha nebo žánr zdroje & Toto je vždy video.\\
\hline
\end{tabular}
\end{table}
\vfill