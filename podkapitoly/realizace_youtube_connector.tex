\section{Realizace YouTube konektoru}
\section{Řešení spolupráce s YouTube API}
\subsection{YouTube Data API (v3)}
\par Tato část rozhraní umožňuje začlenění funkcí YouTube do vlastní aplikace. Proto jí používám pro získání metadat. Nejprve jsem si musel vytvořit google účet a zaregistrovat aplikaci. Pro tento krok slouží \url{https://console.developers.google.com/project}. Každý takto vytvořený projekt má u sebe statistiky s počtem dotazů, počtem chyb, identifikačním řetězcem a názvem. 
\par Po kliknutí na název mého projektu je možné se dozvědět podrobnější informace a změnit konfiguraci projektu. Základní náhled mi poskytuje graf s počtem požadavků, kde vidím jak moc vytěžuji YouTube API. Dále je zde potřeba nechat si vygenerovat unikátní API klíč, pomocí kterého získávám metadata.
\par 
