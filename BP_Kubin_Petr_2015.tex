% options:
% thesis=B bachelor's thesis
% thesis=M master's thesis
% czech thesis in Czech language
% slovak thesis in Slovak language
% english thesis in English language
% hidelinks remove colour boxes around hyperlinks

\documentclass[thesis=B,czech]{FITthesis}[2012/06/26]

\usepackage[utf8]{inputenc} % LaTeX source encoded as UTF-8

\usepackage{filecontents}
\usepackage{csquotes}
\usepackage{multirow}
\usepackage{minted}
\usepackage{longtable}
\usepackage{float}
\usepackage{tabulary}
\usepackage{verbatim}

\usepackage{graphicx} %graphics files inclusion
% \usepackage{amsmath} %advanced maths
% \usepackage{amssymb} %additional math symbols
\usepackage{indentfirst}

\usepackage{dirtree} %directory tree visualisation

%\includeonly{podkapitoly/cil}

% % list of acronyms
% \usepackage[acronym,nonumberlist,toc,numberedsection=autolabel]{glossaries}
% \iflanguage{czech}{\renewcommand*{\acronymname}{Seznam pou{\v z}it{\' y}ch zkratek}}{}
% \makeglossaries

\newcommand{\tg}{\mathop{\mathrm{tg}}} %cesky tangens
\newcommand{\cotg}{\mathop{\mathrm{cotg}}} %cesky cotangens
\newcommand{\note}[1]{{\color{red}\textbf{Připomínka:} #1}}

% % % % % % % % % % % % % % % % % % % % % % % % % % % % % % 
% ODTUD DAL VSE ZMENTE
% % % % % % % % % % % % % % % % % % % % % % % % % % % % % % 

\department{Katedra softwarového inženýrství}
\title{YouTube konektor pro projekt NARRA}
\authorGN{Petr} %(křestní) jméno (jména) autora
\authorFN{Kubín} %příjmení autora
\authorWithDegrees{Petr Kubín} %jméno autora včetně současných akademických titulů
\supervisor{Ing. Petr Pulc}
\acknowledgements{Děkuji vedoucímu práce Ing. Petru Pulci za jeho odborné vedení a své rodině za podporu během celého mého studia.}
\abstractCS{YouTube, snad nejoblíbenější server pro sledování a sdílení videí po celém světě. V mé bakalářské práci se budu zabývat propojením videí na serveru YouTube a jejich popisků, neboli metadat. Takto spárované video a jeho metadata umožní vyhledávačům v projektu OpenNarrative lepší a rychlejší nabízení relevantního obsahu pro další práci s videem, jako například střih. Celý projekt je napsán v jazyce Ruby s využitím již dostupných funkcí z YouTube API.}
\abstractEN{YouTube, perhaps the most popular server to watch and share videos worldwide. In my bachelor thesis I examine the interconnection between videos on YouTube and their labels, or metadata. Thus paired video and its metadata allows search engines in project OpenNarrative better and faster offering relevant content for further work in video, such as editing. The project is written in Ruby using the already available features of YouTube API.}
\placeForDeclarationOfAuthenticity{V~Praze}
\declarationOfAuthenticityOption{4} %volba Prohlášení (číslo 1-6)
\keywordsCS{YouTube API, Ruby, NARRA, OpenNarrative, popis videa metadaty, formát videa H.264}
\keywordsEN{YouTube API, Ruby, NARRA, OpenNarrative, metadata description of the video, H.264 video format}

\begin{document}

% \newacronym{CVUT}{{\v C}VUT}{{\v C}esk{\' e} vysok{\' e} u{\v c}en{\' i} technick{\' e} v Praze}
% \newacronym{FIT}{FIT}{Fakulta informa{\v c}n{\' i}ch technologi{\' i}}

\begin{introduction}
\par YouTube je největší webová stránka pro sdílení videí se sídlem v San Bruno, Kalifornie. Byla založena v únoru 2005 a odkoupena společností google v listopadu následujícího roku. YouTube podporuje WebM a H.264. WebM byl formát pro web společností google. H.264 se používá díky dobré kvalitě komprese a je firemním standartem s hardwarovou podporou. Obsahem YouTube videí může být videoklip, televizní pořad, naučná videa a další. Většina obsahu je tvořena videi od jednotlivců, je zde také možné narazit na soubory od společností například Vevo, nebo BBC.
\par NARRA, projekt centra audiovizuálních studií FAMU, je opensourcová webová služba umožňující společnou práci s audiovizuálním materiálem. Jádro projektu je tvořeno multimedíální databází obsahující metadata jednotlivých médií a informace popisující vazby mezi nimi. Mým cílem je rozšíření projektu NARRA o možnost práce s multimediální službu YouTube pro možnost zpracování médií, neboť YouTube je nejjednodušší způsob uložení médií pro umělce z FAMU, kteří budou software NARRA využívat.
\par Hlavním úkolem je vytvořit konektor umožňující práci s YouTube videi a provázání těchto videí metadaty, které mi poskytne YouTube API. Takto zpracované video pro stažení projektem NARRA bude přístupné pro další střih či úpravy. To povede k usnadnění práce editorů, kteří neznají dodaná data úplně do detailu a potřebují najít ta nejlepší videa pro střih. 
\par Na začátku práce se seznámíte s historií vzniku projektu NARRA, jeho hlavní myšlenkou a použitými technologiemi. V další kapitole zpracuji NoSQL databázi MongoDB, která je součástí projektu NARRA. Dále se teorie posune k YouTube API, jeho možnostem, omezením a technologiím, které mi umožní mou práci dokončit. Uvedu i několik příkladů použití YouTube API spolu s vysvětlením.
\par V další kapitole se seznámíte s teorií testování a testovacím nástrojem RSpec, který je určený pro jazyk Ruby. Poté se teorie přesune k metadatům a DublinCore, kde vysvětlím historii a důležitost popisu elektronického materiálu pomocí metadat. Z teorie tak zbývají pouze dvě kapitoly o vyrovnávací paměti médií, která je potřeba pro uložení videa na server a kodek VP8.
\par V praktické části se dozvíte více příkladů a krátký návod jak pracovat s YouTube API, validaci YouTube URL adresy pomocí regulárního výrazu, identifikaci a uskutečnění přesměrování v rámci HTTP. Dále se dočtete jak dostat z YouTube API informace o videu a jak je uložit ve formě metadat. Na konci práce jsem detailně rozebral teorii a praxi v testování softwaru, který bude součástí většího projektu.
\end{introduction}

\chapter{Cíl práce}
	\par Cílem práce je vytvořit rozšíření pro systém NARRA, které umožní import médií z portálu pro sdílení videí YouTube a~jejich popis metadaty v souladu s DublinCore. Toto rozšíření umožní zpřístupnění videa v systému pro Open Narrative a~dokumentaristé spolu s dalšími tvůrci budou moci využívat YouTube jako své osobní primární úložiště, nebo i zdroj cizího materiálu.
\par Protože systém NARRA potřebuje přístup k multimediálním souborům pro vytvoření náhledů, je naším úkolem také zpřístupnit videosoubor pomocí URL (mezipaměť) pro zpracování systémem. O samotné vytvoření náhledů a~uložení do databáze MongoDB se stará samotný systém, ale základní principy budou vysvětleny i v této práci. V projektu NARRA budou videa dále zpracovávána. Je proto na místě umět nabídnout další videa pro střih, či úpravy, což usnadní práci editorům, kteří neznají dodaná data úplně do detailu a~tak sami netuší, jestli se pro další střih nenabízí něco lepšího.
\chapter{Analýza a návrh}
	\section{Projekt Narra}
\par Narra je projekt s volně dostupným zdrojovým kódem, který se zabývá anotací a propojením audiovizuálních médií a textu. Podobně jako YouTube má také veřejně dostupné API po dokončení souží umělcům, filmařům a dalším, kteří chtějí vytvářet otevřený příběh (open narrative), nebo editovat videa. Dokumentaristé mohou tento projekt využít pro rozsáhlejší díla, díky nabídnutému relevantnímu obsahu s metadaty. Projekt zastřešuje FAMU CAS, které poskytuje práci pro více jak 30 studentů dokončujících Bakalářské a Magisterské studium.
\par S první myšlenkou projektu Narra přišel v roce 2002 - 2003 Eric Rosenzveig a Willy LeMaitre spolu s dalšími mediálními umělci a programátory. Dílo bylo rozdělené na tři části. První "plaListNetWork" byl opensource software vyvinutý na základě konzultací s umělci o audiovizuálním obsahu a rozhraním pro vizualizaci. Umožňoval práci více uživatelů na různých místech a pomocí textových poznámek upravovat popisky skladeb a videí. Druhý "disPlayList" bylo veřejně přístupné rozhraní pro streamování medií z playListNetWork. Jednalo se o webovou aplikaci, která vizualizovala výsledná videa do grafu, ze kterého šel pomocí klíčových slov tvořit další celek. "Ressemblage", neboli poslední část byl výsledkem práce umělců používajících novou technologii práce s médii.
\par V projekt Open Narrative se zapříčinil nejvíce umělec Eric Rosenzveig a reaktor Tomáš Dobruška na FAMU v letech 2010 - 2015. Díky penězům z grantu mohou pokračovat ve vývoji spolu s KSI FIT ČVUT.

\section{Technologie v projektu Narra} 
\par Narra je psaná v jazyce Ruby a poskytuje REST-API pro komunikaci se světem. Další použité technologie jsou Sidekiq, OmniAuth, MongoDB a Rails. Všechny tyto komponenty zajišťují stabilní jádro aplikace, na které je možné připisovat další balíčky (gem), jako v případě mé bakalářské práce. Pro začátek jsem se musel seznámit s databázovám modelem celé aplikace.

\begin{figure}[H]
\includegraphics[width=1\textwidth]{./obrazova_priloha/domain_full.pdf}
\caption{Doménový model celého projektu Narra}
\end{figure}

\par Z celého doménového modelu mě zajímají Meta, MetaItem, MarkMeta a Item.
// todo

\begin{figure}[H]
\includegraphics[width=1\textwidth]{./obrazova_priloha/domain_my.png}
\caption{Doménový model mé části Narry}
\end{figure}
\par // toho popis

	\section {MongoDB}
\par MongoDB\cite{mongodb} je multiplatformní NoSQL databáze. Nepoužívá klasické tabulky, založené na relační struktuře, ale má svá vlastní schémata ve formátu BSON\cite{mongogbinc}. Objektová databáze umožňuje rychlejší a~lehčí integraci dat. Tvůrcem této NoSQL databáze je Americká společnost MongoDB Inc.\cite{mongodb} založená v~roce 2007. V roce 2009 přešla společnost MongoDB\cite{mongodb} k open source řešení a~stala se součástí významných společností, například BOSH.
\subsection {Formát BSON}
\par Formát BSON\cite{mongogbinc} je nejvíce využíván při ukládání a~přenosu dat po síti v~neobjektové databázi. Je velmi podobný již známému JSON formátu. Ukládání dat probíhá v~binární formě, což je efektivnější z hlediska rychlosti i~paměťové náročnosti. V některých případech bude BSON zabírat o něco více místa, neboť potřebuje hlavičku, ve které je uložena délka pole s daty.
Například formát JSON bude uložený v~BSON\cite{mongogbinc}u takto: 

\begin{table}[h]
\begin{tabulary}{\textwidth} {| l | l | l |}
 						& \textbackslash{x}16\textbackslash{x}00\textbackslash{x}00\textbackslash{x}00 								&// total document size \\
 						& \textbackslash{x}02 																						&// 0x02 = type String\\
\{"hello": "world"\} 	& hello\textbackslash{x}00 																					&// field name\\
 						& \textbackslash{x}06\textbackslash{x}00\textbackslash{x}00\textbackslash{x}00world\textbackslash{x}00 		&// field value\\
 						& \textbackslash{x}00 																						&// 0x00 = type EOO \\
\end{tabulary}
\caption[MongoDB příklad formátu BSON]{MongoDB příklad formátu BSON}\label{tab:bson}
\end{table}
\vfill
	\section{YouTube API}
\par YouTube Aplication Programming Interface\cite{apistart} je nástroj pro vývojáře, který umožňuje snadný přístup ke statistikám a~datům na konkrétním YouTube kanálu. API je rozdělené na tři hlavní části:
\begin{itemize}
	\item{Players and Player APIs, které umožňují uživatelům sledování videí ve vaší aplikaci a~zjištění zpětné vazby od uživatelů.}
	\item{Data and Analytics APIs, zběžně popisuje rozhraní pro přístup k~funkcím a~datům na YouTube stránce.}
	\item{Buttons, Widgets, and Tools slouží k~popisu všech nástrojů, které může vývojář použít pro svou aplikaci.}
\end{itemize}

\section{Data and Analytics API}
\subsection{YouTube Data API (v3)}
\par Pro mou práci potřebuji druhou část Data and Analytics APIs\cite{apistart}\cite{apiv3}. Tato část rozhraní umožňuje začlenění funkcí YouTube do vlastní aplikace. Mám v plánu jí použít pro získání metadat, kde http/get požadavkem získám json objekt od YouTube API, s nímž budu nadále pracovat. Pro zahájení práce je potřeba si založit google účet a~svůj vlastní API klíč. Vytvoření projektu a~API klíče popíšu podrobněji v praktické části práce.
\par Celý koncept API umožňuje a~vyžaduje získávání jen potřebných dílčích zdrojů, aby se zabránilo zbytečnému přenosu dat a~nebyla přetěžována síť ani procesor. Tento přístup zajišťuje efektivní práci a~využití prostředků, proto jsou pořadavky rozděleny na části. 
\begin{itemize} 
\item{snippet}
\item{contentDetails}
\item{fileDetails}
\item{player}
\item{processingDetails}
\item{recordingDetails}
\item{statistics}
\item{status}
\item{suggestions}
\item{topicDetails}
\end{itemize}
\par Díky rozkouskování do částí je možné se dotázat na specifické objekty a~ušetřit si tak svou denní kvótu. V návaznosti dojde ke snížení latence a~aplikace bude rychlejší. 

\subsection{Omezení}
\par Každá aplikace má určitá omezení paměti\cite{quota}\cite{googleconsole}, či časového kvanta, které uživateli přidělí. YouTube používá kvóty, ve kterých měří využití výpočetního výkonu pro jednotlivé uživatele. Podporovány jsou čtyři typy operací.
\begin{itemize}
\item {list}
\item {insert}
\item {update}
\item {delete}
\end{itemize}
\par Operace list vrátí GET požadavek spolu žádným či více výsledky. Insert vytvoří pomocí pažadavku POST nový prostředek. Update změní již existující prostředek a~nahradí ho novým. Poslední delete vymaže existující prostředek, který jsme specifikovali. Operace insert, update a~delete vyžadují autorizaci uživatele, nejčastěji pomocí jeho vlastního API\cite{apistart} klíče. Operace list funguje i~v~případech bez autorizace, tak s autorizací.
\par Další důvod pro používání kvót je zajištění jisté úrovně efektivity u vývojářů softwaru používajících Data API, a~ne vytvářet aplikace, které omezují ostatní a~snižují tak kvalitu poskytovaného softwaru. Ceny jednotlivých kvót se pro požadavky liší podle náročnosti operací, která se má provést. Jsou zde dva základní faktory, které ovlivňují z~většiny cenu požadavku. Při načtení ID z~videa zaplatíte 1 jednotku. Operace zápisu stojí přibližně 50 jednotek. Nejdražší je nahrání videa, které se pohybuje okolo 1~600 jednotek za jedno video. To se Vám může zdát jako velmi vysoká cena, ale není tomu tak. 
\par Operace čtení a~zápisu nemají přesně stanovené množství kvót, neboť mohou číst a~zapisovat odlišné množství čístí videí. Pro tento účel YouTube API vytvořilo několik odlišných kategorií, aby umožnilo využít pouze nezbytně malou část kvóty pro požadavek. Po tomto krátkém úvodu se dostáváme k~přidělenému počtu jednotek pro jedno aplikaci. Každá aplikace dostane 50~000~000 jednotek na den, což odpovídá přibližně 1~000~000 operací čtení, kde má každý zdroj dvě části, nebo 50~000 operací zápisu a~450~000 dalších operací čtení, kde má každý zdroj znovu dvě části. Poslední příklad je přibližně 2~000 nahraných videí, 7~000 operací zápisu a~200~000 operací čtení, kde má každý zdroj tři části.
\par Předchozí odstavec byl jen letmý příklad, kolik si YouTube API účtuje jednotek za své služby. Pro detailnější informace je potřeba nahlédnout do své google konzole na adrese \url{https://developers.google.com/youtube/v3/determine_quota_cost}\cite{quota}. Zde jde velmi snadno zjistit konkrétní cenu požadavku aplikace pomocí připravené tabulky. Předběžně vypočítané cena u mé aplikace je 9 jednotek na jeden požadavek, což znamená více jak pět milionů zpracovaných videí za jeden den a~to je více než dostačující kapacita a~proto nemusím omezovat metadata, která v~mé aplikaci plánuji. 
	\section {Testovací nástroj RSpec}
\subsection{Historie}
\par RSpec\cite{davidchelimsky2015} vznikl jako experiment Stevena Bakera, Davida Astelse a~Aslaka Hellesøe. S vývojem začali v roce 2006, na Ruby on Rails. První vydaná verze 1.0 vyšla o~rok později a~obsahovala funkce, které má RSpec dodnes. Měl ovšem několik ne úplně efektivně naimplementovaných částí a~proto musel být přepsán.
\par Koncem roku 2008 zabudoval Chad Humphries "Micronauta", který v sobě obsahoval systém metadat a~poskytoval mnohem lepší flexibilitu, než RSpec 1.0. V roce 2010 začali David a~Chad pracovat na verzi 2. Chtěli celý projekt rozdělit do mudulů, které by pak mohli pužívat samostatně a~navazovat jedním na druhý. Jako jádro posloužil Micronaut, na který navazovali moduly.
\par Listopad roku 2012 byl pro vývor RSpecu zlomový. David Asteles se rozhodl od projektu odpojit a~věnovat se jiným věcem. Lídrem pro RSpec se stal Myron Marston a~pro rspec-rails byl nominován Andy Linderman. Nově vytvořený tým začal pracovat na verzi RSpec 3, která byla spojením a~vyčištěním všech předchozích verzí dohromady. Po vydání RSpec 3 odešel Andy Linderman do důchodu. Dodnes se RSpec rozvíjí díky velké komunitě spolupracovníků.
\par V mé Bakalářské práci budu pracovat s~jádrem testovacího nástroje RSpec a~očekáváními, neboli expectations. V následujících dvou podkapitolách uvedu postup instalace a~ukázky použití daných komponent.

\subsection{RSpec core}
\par RSpec core je samotné jádro aplikace, na které navazují další moduly. Je nezbytnou částí pro testování. Instalace je šložená ze tří příkazů \textit{gem install rspec}; \textit{gem install rspec-core}; \textit{rspec --help} pro nápovědu k nově nainstalovanému softwaru. První příkaz nainstaluje rspec-core, rspec-expectations a~rspec-mocks, což je kompletní balíček pro testování. Druhý příkaz nainstaluje pouze rspec-core, který neobsahuje všechny funkčnosti.
\par Základní struktura testování je velmi podobná hovoru v angličtině. Používají se slova "describe" a~"it", která mají stejný význam jako v mluveném slově.\\
"Describe an order."\\
"It sums the prices of its line items."\\
\begin{minted}{ruby}
RSpec.describe Order do
  it "sums the prices of its line items" do
    order = Order.new
    order.add_entry(LineItem.new(:item => Item.new(
      :price => Money.new(1.11, :USD)
    )))
    order.add_entry(LineItem.new(:item => Item.new(
      :price => Money.new(2.22, :USD),
      :quantity => 2
    )))
    expect(order.total).to eq(Money.new(5.55, :USD))
  end
end
\end{minted}
\par Dále můžeme deklarovat vnořené skupiny pomocí describe, nebo context metod.
\begin{minted}{ruby}
RSpec.describe Order do
  context "with no items" do
    it "behaves one way" do
      # ...
    end
  end

  context "with one item" do
    it "behaves another way" do
      # ...
    end
  end
end
\end{minted}
Další ukázky z rspec-core podrobněji rozeberu v části testování ke konci mé práce.

\subsection{RSpec expectations}
\par Instalace balíčku expectations je naprosto stejná jako instalace rspec-core, ba i jednodušší. Stačí napsat pouze \textit{gem install rspec}, pro použití s~rspec core. V případě testování jinými nástroji, které podporují expectations, je příkaz lehce odlišný \textit{gem install rspec-expectations}. Pro použití ve vývojářském režimu master je potřeba doplnit na začátek souboru tento kus kódu:
\begin{minted}{ruby}
%w[rspec-core rspec-expectations rspec-mocks rspec-support].each do |lib|
  gem lib, :git => "git://github.com/rspec/#{lib}.git", :branch => 'master'
end
\end{minted}
\par Použití je velmi intuitivní, neboť je naprosto shodné s~projevem v angličtině. Velmi hrubá forma je expect(z čeho).operace(s čím). Souvislý kód poté vypadá například takto:
\begin{minted}{ruby}
RSpec.describe Order do
  it "sums the prices of the items in its line items" do
    order = Order.new
    order.add_entry(LineItem.new(:item => Item.new(
      :price => Money.new(1.11, :USD)
    )))
    order.add_entry(LineItem.new(:item => Item.new(
      :price => Money.new(2.22, :USD),
      :quantity => 2
    )))
    expect(order.total).to eq(Money.new(5.55, :USD))
  end
end
\end{minted}
\par Zde máme metodu Order, ve které vytvoříme dvě položky. První má hodnotu (1.1, :USD) a~druhá (2.2, USD), kterou jsme ovšem vytvořili pomocí \textit{:quantity => 2} dvakrát. Proto můžeme otestovat zda součet těchto tří prvků je roven (5.5, :USD). Návratová hodnota testování pomocí expect je true/false. V případě false hlášky oznámí terminál co očekával a~na dalším řádku co dostal od programu. Velmi snadno se tedy pozná kde nastala chyba.

\begin{center}
\begin{longtable}{| p{.40\textwidth} | p{.60\textwidth} |} 
%\caption[RSpec metody testování]{Různé způsoby testů pomocí RSpec}\label{tab:rspec}
\hline
 \textbf{Zabudované komparátory} & \textbf{Význam} \\ 
 \hline\hline
 expect(actual).to eq(exp) & Rovná se ( == ) \\
 \hline
 expect(actual).to eql(exp) & Rovná se ( ? ) \\
 \hline\hline
 \multicolumn{2}{||c||}{Identita}\\
 \hline
 expect(actual).to be(exp) / to equal() & Zda je identické\\
 \hline\hline
 \multicolumn{2}{||c||}{Porovnání}\\
 \hline
 expect(actual).to be(exp) > / < / >= / <= expected & Operace porovnání \\
 \hline\hline
 \multicolumn{2}{||c||}{Regelární výrazy}\\
 \hline
 expect(actual).to match(/exp/) & Zda výraz odpovídá exp \\
 \hline\hline
 \multicolumn{2}{||c||}{Třídy}\\
 \hline\hline
 expect(actual).to be\_an\_instance\_of(exp) & Jestli se aktuální třída == exp \\
 \hline
 expect(actual).to be\_a(exp) & Alias k předchozímu \\
 \hline
 expect(actual).to be\_an(exp) & Alias k předchozímu  \\
 \hline
 expect(actual).to be\_a\_kind\_of(exp) & Alias k předchozímu  \\
 \hline
 \multicolumn{2}{||c||}{Boolovské true / false}\\
 \hline\hline
 expect(actual).to be\_truthy  & Projde když actual != nil OR false\\
 \hline
 expect(actual).to be true    & Projde když actual == true \\
 \hline
 expect(actual).to be\_falsy   & Projde když actual == nil OR false \\
 \hline
 expect(actual).to be false   & Projde když actual == false \\
 \hline
 expect(actual).to be\_nil     & Projde když actual == nil \\
 \hline
 expect(actual).to\_not be\_nil & Projde když actual != nil \\
 \hline
 \multicolumn{2}{||c||}{Očekávání errorů}\\
 \hline\hline
 expect \{ ... \}.to raise\_error & Očekávání, že ... vyvolá error \\
 \hline
 expect \{ ... \}.to raise\_error(ErrorClass) & Očekávání, že ... vyvolá error z ErrorClass\\
 \hline
 expect \{ ... \}.to raise\_error("message") & Očekávání, že ... error bude stejný jako "message" \\
 \hline
 expect \{ ... \}.to raise\_error(ErrorClass, "message")  & Kombinace druhé a~třetí varianty \\
 \hline
 \multicolumn{2}{||c||}{Vyhození chyby}\\
 \hline\hline
 expect \{ ... \}.to throw\_symbol & Očekávání vyhození libovolného symbolu\\ 
 \hline
 expect \{ ... \}.to throw\_symbol(:symbol) & Očekávání vyhození symbolu :symbol\\ 
 \hline
 expect \{ ... \}.to throw\_symbol(:symbol, 'value') & Vyhození symbolu :symbol s~hodnotou 'value'\\ 
 \hline
 \multicolumn{2}{||c||}{Členství v kolekci}\\
 \hline\hline
 expect(actual).to include(expected) & Splněno, když actual obsahuje expected \\
 \hline
 expect(actual).to start\_with(expected) & Actual začíná expected \\
 \hline
 expect(actual).to end\_with(expected) & Actual končí expected \\
 \hline
 \hline
\end{longtable}
\end{center}

%Yielding
%
%expect { |b| 5.tap(&b) }.to yield_control # passes regardless of yielded args
%
%expect { |b| yield_if_true(true, &b) }.to yield_with_no_args # passes only if no args are yielded
%
%expect { |b| 5.tap(&b) }.to yield_with_args(5)
%expect { |b| 5.tap(&b) }.to yield_with_args(Fixnum)
%expect { |b| "a string".tap(&b) }.to yield_with_args(/str/)
%
%expect { |b| [1, 2, 3].each(&b) }.to yield_successive_args(1, 2, 3)
%expect { |b| { :a => 1, :b => 2 }.each(&b) }.to yield_successive_args([:a, 1], [:b, 2])
	\section{DublinCore}
\par DublinCore\cite{dublincoredocementation} je doporučení pro označení metadat, jehož cílem je umožnit rychlé a snadné vyhledávání v elektronických zdrojích. Původně byl vytvořen jako popis metadat v webových stránek, postupně zaujal vyšší instituce a experty z různých odvětví, například muzeí, kni\~ho\~ven a dalších komerčních ogranizací. Vedení sídlí ve státě Ohio v Severní Americe.

\subsection{Historie DublinCore}
\par První setkání tvůrců DublinCore\cite{dublincoredocementation} bylo v březnu 1995. Jejich cílem bylo popsat elektronická data, na základě sémantických pravidel a byla zde uvedena problematika vyhledávání v elektronických dokumentech.

\par Druhý seminář se konal o rok později v dubnu 1996. Tvůrce tentokrát hostilo město Warwick ve Velké Británii. Po přiblížení problematiky mezinárodní komunitě se zaměřili na problém syntaxe a sémantiky, kterou by byly schopné efektivně zpracovat webové aplikace. Rychlé zavedení technologie DublinCore do webových aplikací vedlo k rychlému rozšíření této metodiky do světa. Na tomto setkání byl vytvořen základ architektury metadat: Warwick Framework.

\par V září 1996 byl další seminář přesunut do Spojených států amerických do města Dublin. Hostil převážně experty na grafiku, kteří spolu s tvůrci DublinCore diskutovali o spojení mezi vizuální a textovou částí metadat. Měli za cíl spojit požadavky DublinCore a Warwick Frameworku.

\par Další krok ve vývoji se odehrál v roce 1997 ve městě Canbery v Austrálii. Hlavní myšlenou bylo učinit popis více minimalistický a lépe ho strukturovat. Šlo o zjednodušení a upravení popisu pro následná další rozšíření, neboť každý jistě ví, že na hliněných nohách se kvalitní barák postavit nedá. Proto celé specifikaci dali stejnou strukturu a učinili jí minimalistickou.

\par Další semináře se konaly v říjnu 1997 v Helsinkách, kde byl kladen důraz na datum, působnost a vztah. Ve Washingtonu v roce 1998 bylo svolané setkání, pro sjednocení různých implementací DublinCore. Další seminář byl rekordní v počtu odborníků. Sešlo se jich 120 z 27 zemí. Další semináře už nastíním pouze pomocí tabulky.

\begin{table}[h!]
\centering
\begin{tabular}{| l | l | l |}
\hline
Rok & Město & Země\\
\hline
2000 & Otawa & Kanada\\
\hline
2001 & Tokyo & Japonsko\\
\hline
2002 & Florencie & Itálie\\
\hline
2003 & Seatle & Washington, USA\\
\hline
2004 & Sanghai & Čína\\
\hline
2005 & Madrid & Španělsko\\
\hline
2006 & Manzanillo, Colima & Mexico\\
\hline
2007 & Singapur & Singapur\\
\hline
2008 & Berlin & Německo\\
\hline
2009 & Soul & Jižní Korea\\
\hline
2010 & Pittsburgh & Pennsylvania, USA\\
\hline
2011 & Haag & Nizozemí\\
\hline
2012 & Kuching, Sarawak & Malaysie\\
\hline
2013 & Lisabon & Portugalsko\\
\hline
2014 & Austin & Texas, USA\\
\hline
\end{tabular}
\caption[DublinCore historie]{Schůzky DublinCore mezi lety 2000 - 2014}\label{tab:dublincore}
\end{table}

\subsection{Struktura DublinCore}
\par Pro mou bakalářskou práci jsem zvolil popis sekce 3 z \url{http://dublincore.org/documents/dcmi-terms/}, kterou jsem ještě upravil, abych využil všechna metadata co mi YouTube nabízí a zároveň se zbavil duplicit, které se v této sekci nacházejí. Veškeré popisy jsou dostupné na \url{http://dublincore.org/documents/dcmi-terms/}.
\par Struktura objektu s daty o YouTube videu se lehce lišila od stanovené struktury z DublinCore\cite{dublincoredocementation}. V následující tabulce je náhled na originální změní požadavků, které jsem musel upravit. Více o této úpravě je napsáno v části Realizace, kapitola Popis metadaty.
\hfill
\begin{table}[h!]
\begin{tabular}{|p{.20\textwidth} | p{.40\textwidth} | p{.40\textwidth}|}
\hline
Popisek & Definice & Komentář\\
\hline
Přispěvatel (contributor) & Subjekt zodpovědný za zdroj příspěvku. & Například jméno osoby, organizace, nebo služby.\\
\hline
Krytí (coverage) & Prostorová a časová použitelnost zdroje. & Například označení místa geografickými souřadnicemi, nebo lhůta či časové období.\\
\hline
Tvůrce (creator) & Subjekt zodpovědný za zpřístupnění zdroje. & Například jméno osoby, organizace, nebo služby.\\
\hline
Datum (date) & Doba spojená s událostí v životním cyklu zdroje. & Vyjádření časové informace. \\
\hline
Popis (describtion) & Popis dané entity. & Popis může být textový nebo grafický. \\
\hline
Formát (format) & Formát souboru, fyzický nosič, nebo rozměry zdroje. & Například délka trvání videa. \\
\hline
Identifikátor (identifier) & Primářní klíč zdroje. & Parametr \texttt{v} HTTP GET požadavku na stránku s videem.\\
\hline
Jazyk, řeč & Jazyk zdroje. & YouTube API neposkytuje přímý přístup k jazyku, proto ho nebudu do mých metadat používat.\\
\hline
Vydavatel & Osoba která vydala dílo. & Zde by mohl být vydavatel YouTube, který se dle mého názoru spíše platforma či služba, proto v mých metadatech autor nebude.\\
\hline
Vztah (relation) & Související zdroje. & Tuto informaci YouTube API neposkytuje.\\
\hline
Práva (rights) & Informace o právech obsahu. & Tento atribut je v YouTubeAPI popsán dvěmi entitami. LicensedContent a License.\\
\hline
Zdroj (source) & Přibuzný zdroj od kterého je odvozen popsaný zdroj. & Tento atribut YouTube API neposkytuje.\\
\hline
Předmět (subject) & Téma zdroje. & Toto je znovu entita, kterou YouTubeAPI neobsahuje.\\
\hline
Název (title) & Jméno dané zdroji. & Atribut title, který vracím samostatně kvůli požadavkům na rozraní konektoru.\\
\hline
Typ (type) & Povaha nebo žánr zdroje & Toto je vždy video.\\
\hline
\end{tabular}
\caption[DublinCore metadata]{Ujednocená struktura metadat}\label{tab:dublincore1}
\end{table}
\hfill
	\section{Vyrovnávací paměť médií}
\subsection{Obecný popis}
\par Vyrovnávací paměť\cite{sap} obecně slouží ke zrychlení systému pomocí \uv{na\-ke\-šo\-vá\-ní} informací, které by byly čteny například ze vzdáleného zdroje. Zde je zrychlení docíleno díky různým přístupovým časům mezi vzdáleným diskem a~lokálním diskem. Přístup na vzdálený disk trvá řádově vteřiny, zatímco čtení z lokálního disku trvá o poznání kratší dobu. Velmi rychlou matematikou se dá vypočítat, že je cashe o několik řádů rychlejší v~porovnání se vzdáleným úložištěm, proto je potřeba uložit často používané informace přechodně do vyrovnávací paměti. Mezi nevýhody pamětí cashe patří jejich velikost. Zatímco vzdálený disk či claudové úložiště jsou řádově TB (v případě YouTube nemá šanci jeho velikost smrtelník odhadnout), lokální úložiště je řádově GB \verb|~| TB a~je cenově nákladnější.
\par V projektu NARRA je potřeba pracovat s vyrovnávací pamětí médií, neboť je zde vytěžován server, datové linky a~další komponenty projektu. Řešení spočívá ve vytvoření jednorázového média v~náhledové kvalitě. Náhledová kvalita postačuje pro zjištění obsahu videa a~zároveň rychlou práci pro střih, zatímco po dokončení práce se již vyšlou z projektu příslušné požadavky a~zajistí celé sestříhané video v~nejvyšší možné kvalitě.
\par Uložení náhledového videa má i další důvody: uživateli je možné poskytnout takové zdroje, ke kterým nemá přímo přístup a~dále se video v~náhledové kvalitě neztratí v~případě, že YouTube původní video zablokuje, případně bude odstraněno původním vlastníkem. Tento jev se děje velmi často a~může být spojen s porušováním autorských práv. V případě YouTube je extrémně důležité mít multimédium ve vyrovnávací paměti, neboť není dostupné jako soubor a~tak by bylo třeba ho při každém požadavku znova stahovat a~zpracovávat.

\subsection{Realizace v~projektu NARRA}
\par Výsledná realizace spočívá v~těchto krocích:

\begin{itemize}
\item Můj konektor poskytne informaci, kde se nachází soubor s multimediálním obsahem.
\item Pracovní server NARRA navštíví příslušnou adresu, čímž dojde ke stažení videa, uložení na cestu dostupnou přes http a~zaslání hlavičky 303 s lokací souboru.
\item Pracovní server tedy následuje přesměrování. %(k timeoutu nedochází)
\item Pracovní server přepočítá videosoubor do všech formátů potřebných v~NARRA (WebM ve vysoké a~nízké kvalitě; zvukový soubor ve formátu OGG Vorbis). Odkazy na tyto soubory sám předá do databáze NARRA.
\end{itemize}

\par Uvnitř NARRY se při vytvoření entity Item vytáhnou z konektoru, který umí danou URL obsloužit (existují uvnitř zabudované konektory pro multimediální soubory dostupné přímo přes HTTP), všechny potřebné informace včetně adresy pro stažení fyzického multimediálního souboru. Jako poslední krok po uložení Itemu dojde ke spuštění zpracování. To znamená, že se do fronty úkolů v~systému SideKick zařadí úloha překódování videa do požadovaných formátů.
\par Musel jsem tedy pro moje účely vytvořit způsob, jak stáhnout a~dočasně poskytnout YouTube video jako soubor. Vše je postaveno na serveru nginx. Následující ukázka je pouze návrhem kódu, který poběží na serveru. Podílel jsem se pouze na myšlence tohoto kódu, nikoli na skutečné implementaci. Implementací se zabývali programátoři z projektu NARRA. Z mého konektoru jsem pouze musel zajistit, že bude identifikátor videa v~pořádku.

\begin{minted}{python}
#!/usr/bin/env python
# -*- coding: utf-8 -*-

import web, subprocess, os

urls = ("/.*", "youtube")
app = web.application(urls, globals())

class youtube:
  def GET(self):
    data = web.input(id="0")
    video = data.id
    if video == "0":
       raise web.notfound()
 
    filename = subprocess.check_output(['youtube-dl',
        '--get-filename','-o','"%(id)s.%(ext)s"',video])
    filename = filename.strip('"\n')
    
    os.system("youtube-dl -o /data/%s %s >/dev/null" 
                                    % (filename, video))
 
    raise web.seeother('/'+filename)
 
if __name__ == "__main__":
  web.wsgi.runwsgi = lambda func, 
                 addr=None: web.wsgi.runfcgi(func, addr)
  app.run()
\end{minted}
	\section {Kodek VP8 a ukládání videí v~NARRA}
\par Pro správné pochopení významu kodeku VP8\cite{vp8} je potřeba trocha teorie k WebM\cite{webm}. WebM je otevřený formát multimediálních souborů používaný na webu. WebM soubory se skládají z obrazových toků komprimovaných právě kodekem VP8, nebo VP9. V projektu NARRA je použita komprimace pomocí VP8. V aktuálním nastavení se videa počítají do formátu WebM (video v~kodeku VP8, audio ve Vorbis) ve dvou kvalitách: 720p s bitrate 1Mbps a 180p s bitrate 300kbps. Navíc je počítán čistě zvukový náhled ve Vorbis (kontejner OGG) a pět náhledů v~rozlišení 350x250 ve formátu PNG. 
\par Pro uložení videa se používá kontejner (například WebM, MP4, AVI, MOV, \ldots), tyto formáty definují vnitřní strukturu a složení jednotlivých datových proudů do výsledného souboru. Tento datový proud je potřeba zkomprimovat. K účelu komprimace slouží právě kodek (MPEG, VP8, H.264, \ldots). Komprimace slouží pro zmenšení datového toku pro záznam videa.
\subsection{VP8}
\par Jak již bylo zmíněno VP8 je formát pro kompresi dat vlastněný společností Google. Je založen na knihovně \texttt{libvpx}, která jediná umí zakódovat VP8 video stream. Dekódování probíhá také pomocí Google knihovny \texttt{libvpx}.
\chapter{Realizace}
	\section{Řešení spolupráce s~YouTube API}
\subsection{YouTube Data API (v3)}
\par API v3 umožňuje začlenění funkcí YouTube do vlastní aplikace. Proto jí používám pro získání metadat. Nejprve jsem si musel vytvořit google účet a~zaregistrovat aplikaci. Pro vytvoření google účtu a~zaregistrování slouží \url{https://console.developers.google.com/project}. Každý takto vytvořený projekt má u sebe statistiky s~počtem dotazů, počtem chyb, identifikačním řetězcem a~názvem. 
\par Po kliknutí na název mého projektu je možné se dozvědět podrobnější informace a~změnit konfiguraci projektu. Základní náhled mi poskytuje graf s~počtem požadavků, kde vidím jak moc vytěžuji YouTube API. Dále je zde potřeba nechat si vygenerovat unikátní API klíč, pomocí kterého získávám metadata.

\section{Třída connector}
\subsection{Založení aplikace}
\par Před založením aplikace jsem několikrát navštívil R.U.R. v Praze, kde se projekt NARRA vyvýjí a po vymyšlení mé části aplikace jsem požádal vedoucího práce, aby mi daný projekt předpřipravil, neboť na fork projektu z NARRA jsem neměl dostatečná práva. Pro lehčí kontrolu mého postupu a možnost verzování jsem zvolil \url{www.github.com}. Po předpřipravení projektu podle mého návrhu jsem si na githubu vytvořil účet, vlastní SSH klíč a mohl jsem si celou aplikaci k sobě natáhnout a začít programovat.

\subsection{Validace a inicializace}
\par Vlastní implementace je napsaná v jazyce Ruby a~vývojovém prostředí RubyMine. Pro vyřešení metadat jsem si vytvořil jednu třídu, kterou jsem napojil na narra-core. Dále jsem potřeboval knihovny net/http a json pro snažší práci.
\begin{minted}{ruby}
require 'narra/core'
require 'net/http'
require 'json'

module Narra
  module Youtube
    class Connector < Narra::SPI::Connector
\end{minted}
\par Tímto kusem kódu jsem vytvořil nový modul, který je potomkem Narra::\-SPI::Connector. Další věc jsem musel řešit validaci url. V případě nevalidní url mi stačilo vrátit false, při úspěchu jsem vracel true. Nejdřívě jsem zkoušel do projektu zakomponovat metodu match. Ta ovšem vracela string místo booleovké hodnoty a~proto jsem ji musel nahradit =\~. Takto zkonstruovaný výraz by ovšem nefungoval úplně dokonale, neboť při funkční url by vrátil 0. Stačilo výraz lehce poupravit do tvaru !!(url =\~ RegExp ) a~vše fungovalo jak má.
\begin{minted}{ruby}
!!(url =~ /^(?:http:\/\/|https:\/\/)?(www\.)?(youtu\.be\/|youtube
\.com\/(?:embed\/|v\/|watch\?v=|watch\?.+&v=))((\w|-){6,11})(\S*)
?\$/)
\end{minted}

\par // todo Popis RE

\par Pro správnou funkčnost ověření, zda je url validní či ne, bylo potřeba vyřešit přesměrování. Napčíklad url adresa, která nevypadá ani z části jako validní může vést k youtube videu. Příkladem takové adresy je \url{http://goo.gl/TKMZjS}. Pouhým ověřením přes regulární výraz bych neměl šanci zjistit obsah a validitu odkazu. 
\par Přesměrování vyžadovalo novou knihovnu net/http, ze které jsem použil její zabudované metody. Při vytváření jsem nastavil horní hranici přesměrování na 20, neboť drtivá většina proběhne do tří požadavků. Dále jsem potřeboval zajistit, abych se netočil dokola na několika málo url, což by bylo neefektivní a hloupé. Pro maximum dvaceti prvků mi bohatě postačuje pole s lineárním procházením, neboť hash mapa by byla zbytečný luxus. Takto se mohu podívat zda jsem již nenavštívil nějakou url dvakrát, což by znamenalo zacyklené přesměrování a v mém případě vyhození příslušné vyjímky.
\par První úskalí knihovny net/http nastalo v okamžiku, kdy url neměla v názvu protokol. V tomto případě nebyla schopna rozpoznat server a celý proces zkolaboval. Řešením bylo přidat k url bez protokolu protokol http, který se v případě potřeby přesměruje na https. Kdybych přidal místo pouhého http rovnou https, mohlo by se stát, že některé stránky nebudou fungovat, neboť není zaručené zpětné přesměrování z šifrovaného protokolu na nešifrovaný.
\par Poslední část přesměrování spočívala ve zjištění příslušného kódu, kterým mi stránka odpověděla na get pořadavek. Při kódu 2xx je vše v pořádku a url lze rovnou vrátit, neboť každý jistě ví, že 2xx kód je symbolem úspěchu. Číslovka začínající trojkou je ovšem zajímavějším protože se jedná o přesměrování. V tomto případě musí programátor zjistit, na kterou stránku se dostane a proces opakovat, než dostane kód 2xx, nebo než zjistí, že je v cyklu. Poslední skupina jsou kódy 4xx a 5xx značící chybu klienta a chybu serveru.

\par Po zjištění, zda je požadovaná url adresa validní, bylo potřeba ještě provést inicializaci. Zde vytáhnu z YouTube API všechny potřebné informace o videu, které budu dále zpracovávat. Zde je velké množství parametrů, které YouTube API nabízí k výběru. Dále je potřeba vygenerovaný API klíč, pomocí něhož YouTube pozná, komu ubrat denní kvótu za požadavek. V následující tabulce je příklad parametrů pro API.

\begin{table}[h!]
\centering
\caption[Parametry YouTube API a jejich význam]{Parametry YouTube API a jejich význam}\label{tab:apiparams}
\begin{tabular}{| l | l |}
\hline
Parametr & Význam \\
\hline
snippet & Zobrazí hlavní informace o videu \\
\hline
contentDetails & Zobrazí detaily obsahu \\
\hline
fileDetails & Zobrazí detaily souboru \\
\hline
player & Zobrazí detaily přehrávače \\
\hline
processingDetails & Zobrazí podrobnosti zpracování \\
\hline
recordingDetails & Zobrazí podrobnosti nahrávání \\
\hline
statistics & Zobrazí statistiky videa \\
\hline
status & Zobrazí status \\
\hline
suggestions & Zobrazí návrhy \\
\hline
topicDetails & Zobrazí detaily o tématu videa \\
\hline
\end{tabular}
\end{table}

\par Konrétní požavavek byl na adresu \url{https://www.googleapis.com/youtube/v3/videos?id=#{@videoid}&key=klíč&part=část}, kde @videoid je inicializovaná hodnota pro identifikátor videa, klíč je unikátní API klíč programátora a část je jeden, nebo několik parametrů oddělených čárkou. Zde je nutné vzít v potaz, že za vytížení YouTube serveru se platí a v jistých případech nemalou částí denní přidělené kvóty jednotek. Detailní výpočet jsem popsal již v kapitole o YouTube API, zde se o tomto omezení zmiňuji podruhé, neboť jsem nepoužil všech deset parametrů, ale jen čtyři.
\par Snippet, který napíše o videu většinu informací. ContentDetails byl potřeba pro splnění pořadavků z DublinCore, statistics přidají do obsahu počty sledovaností a~status zobrazí licenci a~informace o sdílení videa. Tyto čtyři položky stačí pro pořadovaná metadata zadavatelem a~při přidání dalších bych zbytečně omezoval maximální počet vrácených položek díky omezení YouTube API a aplikace by pozbývala na efektivitě.
\par Při inicializaci je druhý parametr klíč, který slouží 


\par Nyní již mám k dispozici json objekt a můžu se pustit do práce. První pokus o rozparsování proběhl ručně. Vždy jsem si pomocí methody split rozdělil objekt na pole o dvou částech a druhý index jsem rozdělil znovu podle čárky a odřádkování. Pro lepší představu o vizuální stránce kódu sem dám ukázku vytažení obsahu title.
\begin{minted}{ruby}
pom = @youtube_json_object_snippet.split('"title": "')[1]
@name = pom.split("\",\n")[0]
\end{minted}
\par Toto řešení bylo vcelku jednoduché, rozdělení podle ",\verb|\|n bylo v pořádku, neboť YouTube v popiscích provedlo escapacování těchto znaků a nemohlo dojít k nechtěnnému rozdělení ve špatném místě. Kód ovšem vypadal naprosto hrozně a proto jsem zvolil již hotovou variantu json parseru. Pro porovnání ukázka kódu s knihovnou json.
\begin{minted}{ruby}
my_hash = JSON.parse(@youtube)
my_hash["items"][0]["snippet"]["title"]
\end{minted}

\par Toto řešení je mnohem přehlednější a další programátor má usnadněné pochopení vnitřní struktury json objektu. Po této odbočce se dostáváme zpátky k řešení, kde druhým zmiňovaným způsobem vracím název videa samostatně a ne v komplexní struktuře metadat. Tato alternativa byla zvolena záměrně díky ukládání videí v mateřském projektu. Dalším požadavkem mateřkého projektu bylo vrácení typu videa :video. Tím jsem měl za sebou základní část a mohl jsem pokračovat s metadaty.


\subsection{Metadata}

\par Pro popis metadaty jsem vycházel ze struktury DublinCore, která ovšem ne úplně 100\% odpovídala mé představě a představě youtube vývojářů a proto jsem celou kostru musel upravit. 

\begin{table}[!ht]
\centering
\caption[Metadata předávaná do NARRA]{Metadata předávaná do NARRA}\label{tab:bson}
\begin{tabular}{| p{.30\textwidth} | p{.70\textwidth} |}
\hline
	Název & Obsah \\
\hline
\hline
	videoId & Jednoznačný identifikátor videa. \\
\hline
	channelId & Jednoznačný identifikátor kanálu, pod kterým je video k dispozici. \\
\hline
	channelTitle & Název kanálu, pod kterým je video k dispozici. \\
\hline
	publishedAt & Přesný čas vydání a zvěřejnění videa. \\
\hline
	description & Popis k videu. \\
\hline
	categoryId & Číslo kategorie, do které patří dané video. \\
\hline
	liveBroadcastContent & Booleovská hodnota, zda je obsah ve videu vysílaný živě. \\
\hline
	viewCount & Počet shlédnutí videa. \\
\hline
	likeCount & Počet udělení líbí se. \\
\hline
	dislikeCount & Počet udělení nelíbí se. \\
\hline
	favouriteCount & Počet přidání do oblíbených. \\
\hline
	commentCount & Počet komentářů k videu. \\
\hline
	duration & Čas trvání. \\
\hline
	dimension & Zda je video 2d, nebo 3d. \\
\hline
	definition & Jaké je maximální možné rozlišení videa. \\
\hline
	caption & Booleovská hodnota, zda video obsahuje či neobsahuje titulky. \\
\hline
	licensedContent & Zda obsah videa podléhá licencování. \\
\hline
	regionRestriction & Zda je video zakázané v nějaká zemi. \\
\hline
	uploadStatus & Zda je nahrané video již kompletní, či ještě ne. \\
\hline
	privacyStatus & Informace o soukromí videa. \\
\hline
	license & Kdo vlastní licenci k videu. \\
\hline
	embeddable & Zda je možné toto video použít k vložení. \\
\hline
	publicStatsViewable & Booleovská hodnota o zobrazitelnosti veřejných statistik. \\
\hline
	timestamp & Čas ve formátu utc, kdy byla metadata pořízena. \\
\hline
\end{tabular}
\end{table}

\par Jak jste si mohli povšimnout v této tabulce mi chybí titulek videa. Toto řešení je součástí návrhu, kde vracím název videa samostatně pro lepší následné ukládání v mateřském projektu. Dále mám staticky zadefinovaný typ videa, který se nemění. 
\par Pro správné pochopení, jak extrhahovat metadata ze struktury JSON objektu je potřeba zjistit jak přesně vypadá. Celý objekt je jeden prvek obsahující pole items, ze kterého používám nultý prvek. V této úrovni rozhoduji, zda vyberu data z položky snippet, statistics, contentDetails nebo status. Po zvolení například snippet se dostanu o úroveň hlouběji a mohu vybrat konkrétní položku, například channelId. Stejným způsoben jsou dostupná všechna metadata z JSON objektu, pouze u restrikce zemí vzacím složitější strukturu než string.
\par Poslední prvek metadat timestamp nenajdu v DublinCore ani v YouTube API, je ovšem důležité ho do dat zařadit, kvůli udržovatelnosti. V mateřské aplikaci bude nejspíš také časový otisk, který není v této chvíli ještě dodělán a proto jsem ho umístil do metadat. Je zde kvůli kontrole, jak staré jsou statistiky u videa a například pro automatizovanou kontrolu metadat starších než dva týdny se tento údaj hodí. Ještě jsem přemýšlel zda bude dobré řešení vložit časový otisk přímo do metadat, nebo raději mimo, ale řešení s otiskem v metadatech vyhrálo. Nejpádnější důvod byl, že při aktualizaci se zavolá pouze metoda metadata a nebude se muset volat žádná další, neboť znovu stahovat video nemá cenu, v případě přidání titulků se zavolá ještě download\_subtitles. Také jméno, identifikátor, url a typ videa se nebudou měnit.
\par Celkem můj gem poskytuje k jednomu videu 24 metadat a odděleně také jméno videa, typ videa. V součtu se jedná o dvacet šest položek, které umožní rychlejší vyhledávání a relevantní obsah pro každého uživatele, který moje rozšíření využije.

\subsection{Dokončení}
\par Na závěr mi zbývalo vrátit youtube url ve formátu, kde bude pouze video stream bez ostatních elementů, neboť ze standartní url by to bylo moc práce navíc pro jádro aplikace. Pro tento účel souží adresa \url{https://www.youtube.com/v/}, za kterou stačí přiřadit identifikátor videa a požadovaný video stream je na světe. 
\par Poslední kus kódu patřil stažení titulek. Na první pohled se to zdálo jako velmi jednoduchý úkol, ovšem stažení titulek stojí 200 jednotek. Proto je potřeba autentizace API klíčem. Tímto klíčem je potřeba být přihlášen již v jádru aplikace při spuštění a na můj konektor se jen dotázat na stažení titulek. Jelikož je ještě autentizace pomocí OAuth v mateřském projektu nedořešená, předpřipravil jsem můj kus pro stažení titulek pouze pro aktuální funkčnost, která zabezpečí, že po autentizaci pomocí OAuth začne mé stažení titulek fungovat.
\par Titulky k videu jsou k dispozici z \url{https://www.googleapis.com/youtube/v3/captions/}, za kterou se opět přiřadí identifikátor videa. Bez autorizace ovšem nahlásí stránka chybový kód: "Login Required".

	\section{Testování YouTube konektoru}
\subsection{Teorie testování}
\par Pro správné pochopení teorie testování\cite{si1} si musíme uvědomit, že pomocí testů prokážeme že software obsahuje chyby při nesplnění testu. Při splnění všech testů nemůžeme dokázat, že je software stoprocentně bez chyb, neboť může existovat chyba, kterou testy neodhalily. Proto je potřeba se důkladně věnovat testování, abychom snížili riziko chyby na nejnižší možnou úroveň.
\subsection{Jednotkové testy}
\par První druh testů jsou jednotkové testy\cite{si1}. Ty se zabývají konkrétní funkčnotí jednotlivých method a tříd. Testování probíhá pro jednu konkrétní třídu, obvykle běží krátkou dobu a testování probíhá na lokálním počítači vývojáře. Tyto testy pokrává v Ruby nástroj RSpec, který jsem popsal na začátku práce. V mých testech jsem se zaměřil na zjištění správné odezvy method a získání správných výsledků. 
\subsection{Integrační testy}
\par Integrační testování\cite{si1} se provádí po dokončení jednotkových testů a slouží k bezchybnému začlenění nového kusu aplikace do stávajícího projektu. Těmito testy se ověřuje nejen integrace komponentů, ale i spolupráce se seftwarem případně hardwarem při složitějších projektech, které potřebují specifický hardware nebo software. Testování probíhá začleněním jedné z komponent a postupném přidávání dalších komponent k celému projektu. Integrační testování je často zanedbáváno díky rozpočtu projektu. Takto nezachycené chyby se ovšem projeví v dalších testech a je proto dobré integrační testy nepřeskočit.
\subsection{Systémové testy}
\par V pořadí již třetí zkouška funkčnosti projektu se zabývá chováním celého projektu jako celku. Provádí se zde simulace scénářů a kroků, které mohou po nasazení projektu nastat. Je to poslední testování před předáním projektu zákazníkovi a proto obvykle probíhá několikrát. Systémové testování\cite{si1} je nezbytné pro výstupní kontrolu projektu.
\subsection{Akceptační testy}
\par Akceptační testy jsou prováděné zákazníkem spolu se zástupcem firmy na předem dohodnutých scénářích. Jedná se o otestování, zda zadavatel správně pochopil požadavky zákazníka. Odhalení závažných nedostatků v této poslední fázi bývá nejhorším možným scénářem pro vývojáře, který musí následně aplikaci předělávat. Můj gem jsem testoval jednotkovými a integračními testy, neboť systémové a akceptační testování zajišťují programátoři jádra projektu.


\subsection{Testování třídy Connector}
\subsubsection{Testování url}
\par První část testu kontroluje vytváření a validaci url. Před první částí testů jsem si nastavil testovanou url a zjistil zda se správně vytvoří objekt dané třídy. Poté jsem zkontroloval tři globální proměnné zda se nezměnil jejich obsah. Tyto dva testy zajistili primární funkčnost konektoru a mohl jsem zkontrolovat url. Zde jsem hledal různé url adresy, které nikam nepřesměrují a jsou validní. Dále jsem potřeboval najít nevalidní url, což stačilo vhodným způsobem modifikovat validní. Dále jsem potřeboval najít url, která se přesměruje na validní. Zde jsem použil funkci youtube, kde je možné zvolit sdílení videa, což vytvoří url, která se přesměruje. Poté jsem ještě použil google zkracovač, což je také velmi dobrá methoda pro ověření validace a přesměrování.
\subsubsection{Testování objektu konektoru}
\par Další testy již probíhali pro určitý objekt konektoru. Zde jsem opět našel několik youtube videí na kterých jsem testoval příslušná metadata. Pro hezčí zápis testu jsem musel zápis metadat předělat z \textit{\{name:'channelId', value:'\#\{@channelId\}'\}} do zápisu \textit{\{'channelId'=>'\#\{@channelId\}'\}}, což mi umožnilo lepší a přehlednější přístup k testování metadat. Url adresy jsem volil pro pokrytí speciálních znaků v popiscích, pro neobvyklá metadata, které jsou u videí jen zřídka a pro pokrytí všech metadat. Zde byl problém s proměnlivými statistikami u videa, kde například u počtu shlédnutí jsem musel při testování stále upravovat testovanou hodnotu. Proto jsem vizuálně otestoval s JSON objektem shodu v proměnlivých počtech u statistik a poté jsem test zakomentoval.
\par V nástroji RSpec je v bloku before nekolik možností parametrů v závorkách. V první části testování, kdy jsem měl ještě málo testů, trval celý blok okolo deseti vteřin. V závěrečné fázi práce, kdy jsem otestoval vše co mě napadlo trval test i přes jednu minutu. To se mi nezdálo a proto jsem se důkladně zaměřil na parametrizaci v bloku before. Při parametru :each, který jsem měl ze začátku se celá inicializace provedla před kazdým blokem it, proto testy trvaly tak nehorázně dlouhou dobu. Po změně na :all se rázem čas testů dostal stabilně do deseti vteřin. Proto je potřeba dávat velký pozor, zda neděláme v programu stejné věci vícekrát.
\subsubsection{Testy na videích}
\par Pro první testované video jsem zvolil "Zimní montáže - 1. díl" z kanálu mrk.cz. Zde byl zajímavý prvek ve formě odřádkování v popisu videa. Druhé testované video bylo znovu v duchu rybolovu s zajímavým popiskem, který byl tentokrát prázdný. Dále jsem chtěl na druhém videu otestovat živé vysílání neboť toto video bylo vysíláno živě a přesto je parametr liveBroadcastContent roven hodnotě false, neboť ve chvíli testování má již video statický obsah a již není vysíláno živě.
\par Dále jsem vytvořil jeden test pro vyzkoušení titulků. Zde jsem potřeboval ověřit zda mi program správě vyhodí vyjímku při neexistujících titulcích. Tato vyjímka je velmi opodstatněná neboť každý download titulků stojí 200 jednotek z youtube api kvóty a bylo by nežádoucí nechat uživatele několikrát zkusit stáhnout neexistující titulky. Další test spočíval ve správném zpracování dlouhé url adresy pomocí regulárního výrazu při validaci url a pro zkoušku vytažení všech dostupných dat.
\par Páté testované video bylo vytvoření kvůli možnosti stáhnout validní titulky k videu. Další test vyzkoušel časovou známku u titulků pro čas vytvoření. Blok číslo deset má na starost otestování možnosti stáhnout video, neboli dostat se na url, která obsahuje pouze video stream. Předposlední test znovu zkusil stažení titulků a oveření zda bude vyhozena vyjímka u neexistujících titulek. Závěrečný test je zde z důvodu omezení videí v různých zemích. Zde youtube dává hodnoty blockedIn v případě, že je video zakázáno v zemi, kterou vrací v poli. V případě, že video není nikde blokováno tento atribut zmizí. Znovu jsem proto musel otestovat správné reagování na omezení.
\subsubsection{Shrnutí jednotkových testů}
\par V jednotkových testech jsem se snažil pokrýt všechny možné nástrahy, které mi je schopen uživatel nadělit a otestovat, zda můj program reaguje podle předpokladů. Všechny testy proběhly bez chyb a proto jsem snížil pravděpodobnost že můj software obsahuje chyby na velmi malou. S otestovaným softwarem pomocí jednotkových testů mohu pokračovat v integračních a systémových testech s mateřskou aplikací.
\subsection{Integrační testy}
\par Integrační testování odhalilo několik ošetření, které je potřeba provádět v mém balíčku a nespoléhat na jádro aplikace. Jednalo se o zařazení metadat, která by měla hodnotu nil. V takovém případě jsem nesměl přidat danou položku s metadaty k výslednému videu, neboť ošetřování obsahu metadat při zpracovávání videa by bylo zbytečné zdržování výpočetních prostředků serveru. Další chybu integrační testy neodhalily a má část aplikace fungovala dle očekávání.
\begin{conclusion}
	\input{podkapitoly/zaver.tex}
\end{conclusion}

\bibliographystyle{csn690}
\bibliography{mybibliographyfile}

\appendix

\chapter{Seznam použitých zkratek}
% \printglossaries
\begin{description}
	\item[FAMU] Filmová a televizní fakulta Akademie múzických umění
	\item[CAS] Centrum audiovizuálních studií
	\item[KSI] Katedra softwarového inženýrství
	\item[FIT] Fakulta informačních technologií
	\item[ČVUT] České vysoké učení technické v Praze
	\item[MU] Masarykova univerzita
	\item[API] Programovací rozhraní aplikace
	\item[EOO] End of object(Konec objektu)
	\item[DSL] Doménově specifický jazyk
\end{description}


\chapter{Obsah přiloženého CD}

\begin{figure}
	\dirtree{%
		.1 narra-youtube\DTcomment{složka s implementační částí}.
		.2 lib/narra\DTcomment{konektor NARRA}.
		.2 spec\DTcomment{testování}.
		.1 zadani.txt\DTcomment{zadání práce}.
		.1 thesis\DTcomment{složka se textovou částí práce}.
		.2 BP\_Kubin\_Petr\_2015.pdf\DTcomment{text práce ve formátu PDF}.
		.2 BP\_Kubin\_Petr\_2015.tex\DTcomment{zdrojová forma práce ve formátu \LaTeX{}}.
	}
\end{figure}

\end{document}
